\documentclass[11pt, a4paper]{notunprofessional}
% Specific options: teal - changes color used for titles.
% Other options passed to article class.
\begin{document}

\nupprofile
  {Bertrand Russell}
  {Mathematician | Philosopher}
  {British mathematician, philosopher, logician, and public intellectual.
  Considerable influence on mathematics, logic, set theory, linguistics,
  artificial intelligence, cognitive science, computer science and various
  areas of analytic philosophy, especially philosophy of mathematics,
  philosophy of language, epistemology, and metaphysics.}

\nupcontact
  {Cambridge, UK}
  {bertie@cambridge.ac.uk}
  {+44 712 345 67 89}
  {bertie}

\nupskills{
  Philosophy/5,
  Maths/4,
  Logic/4
}

\nuplanguages{
  English/5,
  Français/3,
  Deutsch/2
}

\nupfirstpage{
  \nuptitle{Mathematics}
  \begin{itemize}
    \item Published \emph{The Principles of Mathematics} in 1903.
  \end{itemize}
  \nuptitle{Philosophy}
  \begin{itemize}
    \item Helped develop what is now called \emph{Analytic Philosophy}.
    \item Greatly influenced modern mathematical logic.
    \item Made language a central part of philosophy.
  \end{itemize}
}

\nupexperiencesection{
  \nupexperience{1910 --- 1916}{Cambridge University}{Lecturer}{
    He was considered for a Fellowship, which would give him a vote in the
    college government and protect him from being fired for his opinions, but
    was passed over because he was "anti-clerical", essentially because he was
    agnostic. He was approached by the Austrian engineering student Ludwig
    Wittgenstein, who became his PhD student.
  }
  \nupexperience{1918}{Brixton Prison}{Convict}{
    "I found prison in many ways quite agreeable. I had no engagements, no
    difficult decisions to make, no fear of callers, no interruptions to my
    work."
  }
}

\nupeducationsection{
  \nupeducation{1893}{Cambridge University}{B.A. Mathematics}{
    Graduated as 7\textsuperscript{th} wrangler.
  }
}

\makecv

\end{document}
